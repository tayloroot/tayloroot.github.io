\logosection{\faGraduationCap}{教育背景}
\datedline{\textbf{华中科技大学(985)}, 机器人方向,\textit{ 工学硕士(推免)}}{\dateRange{2017.09}{2020.06}}
\begin{itemize}[topsep=0pt, left=1em]
  \item \textbf{研究内容: } 点云特征描述,目标检测,点云配准
  \item 排名6/233(3\%),SCI论文1篇,EI论文2篇,专利5项,软著1项,国家级竞赛获奖2次
  \item \textbf{关键课程: } 矩阵论,数理统计,数值方法,工业图像处理,算法设计与分析,数字几何处理,机器学习等 % 多视图几何
\end{itemize}

\datedline{\textbf{北京科技大学(211)}, 机械电子工程,\textit{ 工学学士}}{\dateRange{2013.09}{2017.06}}
\begin{itemize}[topsep=0pt, left=1em]
  \item 排名5/235(2\%),国家奖学金2次,国家励志奖学金1次,国家级竞赛获奖5次
  %\item \textbf{关键课程: } 工科数学分析,概率论,线性代数,C++程序设计,数据结构与算法等
\end{itemize}

\logosection{\faCogs}{专业技能}

\begin{itemize}[parsep=0.5ex, label=\ding{70}]
  \item \textbf{编程语言}: 
  C++(Windows+VS, Linux+CMake), Python, MATLAB
  \item \textbf{英语}: 
  CET-6(510),良好的英文阅读与写作能力
  \item \textbf{算法库/软件}: 
  PCL, OpenCV, CGAL, Open3D, Blender, SolidWords, AutoCAD
\end{itemize}

\logosection{\faSuitcase}{工作经历}

\datedline{\textbf{杭州海康威视研究院 | HRI},智能传感算法工程师}{{2020.07 -}{ 2024.07}}

\begin{itemize}[topsep=0pt, left=1em] % 调整 left 以改变缩进,label 更改项目符号
  \item \textbf{工作内容:} 
  1.点云处理(滤波去噪、点云着色、点云语义化);
  2.面片处理(表面重建、矢量化建模、纹理映射);
  3.多传感器间标定(相机、激光雷达、IMU);
  4.数字图像处理;
  5.点云/mesh模型漫游渲染等。
\end{itemize}

\logosection{\faWrench}{项目经历}
\datedline{\textbf{3D勘测扫描仪}, 点云语义矢量建模技术平台开发}{\dateRange{2021.06}{2024.05}}
\begin{itemize}[topsep=0pt, left=1em, itemsep=1.5pt]
  \item \textbf{多传感器标定(独立完成):}
  相机、雷达和IMU的内外参/时延标定,标定精度<(5mm,0.3°),时延波动<1ms。
  \item \textbf{点云着色(独立完成):}
  将场景点云拆解为平面与非平面,对于平面,通过体素划分使其规则化,
  利用视点特征构建MRF能量函数,优化求取着色图像;对于非平面,认为观测颜色满足高斯分布,
  利用极大似然估计其最适颜色。对比竞品(欧思徕/飞马),所提算法在清晰度/准确度/完整度方面均占优。
  \item \textbf{点云语义分类(主导,跨部门开发):}
  负责方案调研、可行性分析和选型;实施过程中负责点云-视点特征模块开发;
  最终通过多视点聚合策略,融合点云与图像特征,将关键元素mIoU由72.3\%提升至89.7\%。
  \item \textbf{矢量化建模(联合开发):}
  主要负责墙体矢量重建,将其建模为\{0,1\}整数规划问题。
  对比基线方案\href{https://inria.hal.science/hal-02924409}{(KSR)},在满足精度约束的前提下,完整度由71\%提升至95\%,
  耗时由>1h降低为<10min@单层普通室内(500平)。
  \item \textbf{实景/虚拟纹理映射(独立完成):}
  将面片的图像选择问题建模为多标签的MRF能量最小化问题,
  并且以平面为基元进行纹理生成(补全/拼接/消缝等)。
  对比基线方案\href{https://github.com/nmoehrle/mvs-texturing}{(MVE)},缓解了纹理拉伸/接缝/错位/缺失等现象。
\end{itemize}

\datedline{\textbf{3D测距相机}, 基于无人机扫描的野外场景建模模块开发}{\dateRange{2023.06}{2023.11}}
\begin{itemize}[topsep=0pt, left=1em]
  \item \textbf{场景建模(独立完成):}
  1.地面提取(CSF算法)、道路/桥梁提取(区域生长算法)、
  表面重建(Delaunary三角化和多边形三角剖分)以及深度图生成;
  2. 输电线的参数化方程拟合。对比友商,
  精度由63\%@0.1m提升至91\%@0.1m, 完整度由75\%@0.3m提升至99\%@0.3m,
  耗时由>30min提升至<3min@20万平。
\end{itemize}

\datedline{\textbf{基于RGB-D相机的虚拟试衣镜原型系统}, 人体模型生成模块优化}{\dateRange{2020.10}{2021.05}}
\begin{itemize}[topsep=0pt, left=1em]
  \item \textbf{人体姿态优化(独立完成):}
  完成\href{https://smpl.is.tue.mpg.de/}{SMPL}人体关节点和深度图像人体关节点的非刚性对齐,
  将其转化为最小二乘优化问题进行求解。对比基线方案\href{https://github.com/mkocabas/VIBE}{(VIBE)},
  消除了人体模型的漂浮感和抖动。
\end{itemize}

\datedline{\textbf{世界机器人大会-机器人双臂协作抓取挑战赛}, 视觉检测模块开发}{\dateRange{2018.01}{2019.09}}
\begin{itemize}[topsep=0pt, left=1em]
  \item \textbf{视觉检测模块(独立完成):}
  基于Kinect-V2采集的RGB-D场景数据,
  完成目标识别/定位,堆叠检测以及障碍物检测,
  机器人抓取成功率100\%。获2018, 2019该赛事冠军。 
\end{itemize}

\logosection{\faTrophy}{主要论文/竞赛获奖}
\begin{itemize}[parsep=0.5ex, label=\ding{70}]
  \item HoPPF: A novel local surface descriptor for 3D object recognition, 
  \textit{Pattern Recognition, 2020}. Huan Zhao, \textbf{Minjie Tang*}, 
  Han Ding(导师一作,SCI,IF:8.0,JCR分区:Q1,他引:79)
\href{https://www.sciencedirect.com/science/article/abs/pii/S0031320320300777}{[论文链接]} 
  \item 二进制点云局部特征描述子研究, \textit{机械工程学报, 2020}. \textbf{唐敏杰}, 赵欢*,丁汉(EI,他引:17)\href{https://s.wanfangdata.com.cn/paper?q=%E4%BD%9C%E8%80%85%3A%22%E5%94%90%E6%95%8F%E6%9D%B0%22%20%E4%BD%9C%E8%80%85%E5%8D%95%E4%BD%8D%3A%20%22%E5%8D%8E%E4%B8%AD%E7%A7%91%E6%8A%80%E5%A4%A7%E5%AD%A6%22}{[万方数据库]} 
%  \item 面向动作识别的旋转投影变换关节特征骨架描述, \textit{机械工程学报(2020)}, 巫晓康, 赵欢*,唐敏杰,丁汉(EI,他引:1)
  \item 2018、2019年世界机器人大会—共融机器人抓取挑战赛\textbf{冠军}(第二完成人,视觉负责人)\href{https://www.worldrobotconference.com/cn/view/1129.html}{[官网地址]}  
  \item 第六届全国大学生数学竞赛(非数学类)北京赛区\textbf{一等奖}, 2015年12月
\end{itemize}

%\logosection{\faInfo}{个人总结}
%\begin{itemize}[parsep=0.5ex]
%  \item 擅长并热衷于传统的三维数据处理(点云/Mesh等),对前沿的NERF与3DGS相关技术也有所涉猎。
%\end{itemize}

%\logosection{\faHeart}{获奖情况}
%\datedline{NeurIPS Best Paper Award}{2099.12}



% increase linespacing [parsep=0.5ex]

%%%% 如果多页简历,可以手动在适当位置插入 \newpage 或者 \clearpage 开始新一页
